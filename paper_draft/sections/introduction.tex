\section{Introduction}

Random walks serve as fundamental objects of study in both cryptography and game theory, albeit from different perspectives. In cryptography, random walks often appear in the analysis of security protocols and sampling algorithms \cite{mitzenmacher2017probability}, while in game theory, they emerge naturally in the study of repeated games and learning dynamics \cite{young2004strategic}.

This paper develops a unified framework that bridges these perspectives, demonstrating how security properties in cryptography can be reinterpreted through game-theoretic concepts. Our approach reveals deep connections between cryptographic security games and repeated games with random walk dynamics, leading to novel insights in both fields.

Our main contributions are:
\begin{itemize}
    \item A formal framework unifying cryptographic security games and game-theoretic random walks
    \item Three fundamental theorems establishing equivalences between security properties and game-theoretic equilibria
    \item Applications to commitment schemes and zero-knowledge proofs
    \item New algorithmic techniques for analyzing security protocols using game-theoretic methods
\end{itemize}